
Para obter os benefícios de um modelo de dados baseado em grafos, é de
interesse usar de ferramentas que melhor aproximam esse comportamento. Detalhes
adicionais como propriedades, \emph{labels}, direção dos arcos e significados
semânticos podem ser usados em modelos mais específicos, como RDF, hipergrafos
ou grafos com propriedades. Existem duas categorias de Bancos de Dados de
Grafos que são de grande interesse e uso atualmente:
\emph{SGBD de Grafos} e \emph{Graph Analytics Systems}.

\emph{Sistemas de Bancos de Dados de Grafos} enfatizam armazenamento
persistente e garantias transacionais em um ambiente multi-usuário, com esquema
flexível e tipicamente com uma linguagem própria de consulta, que permita
expressar transações depedentes da estrutura do grafo apropriadamente. A
independência lógica e física entre os dados é conquistada pela adoção de um
modelo de alto nível do grafo, por exemplo um grafo com propriedades, aliado a
uma separação do armazenamento físico. Esses modelos podem ser classificados
como nativos (ou seja, com armazenamento também em formato de grafo) ou
não-nativos (usando de outra forma de armazenamento, por exemplo documentos).
Exemplos incluem Neo4j, OrientDB, ArangoDB, Gaffer, SAP Hana Graph, e JanusGraph.

\emph{Graph Analytics Systems} ou \emph{Graph Processing Frameworks} focam-se
principalmente em tarefas analíticas sobre grandes grafos, com tolerância a
falhas e processamento distribuído. Essas tarefas podem demorar horas e ser
realizadas sobre um largo \emph{cluster}, como por exemplo um algoritmo de
\emph{PageRank} ou centralidade. São adotados processamentos assíncronos e
distribuídos, e se busca reduzir o I/O do cluster, no qual assume-se que não
haja recursos compartilhados e qualquer transmissão de dados ou informações
gerará um custo. Exemplos desse tipo de ferramenta incluem Pregel, Blogel,
Apache Giraph e Gradoop.

A principal diferença entre essas duas classes de ferramentas é no tipo de
transação a que dão mais importância e, por outro lado, onde perdas são mais
aceitáveis. GDBMS focam-se na garantia de transações de \emph{update}, afetando
tipicamente apenas uma porção pequena do grafo, e pagam o custo em tempo e
eficiência para essa garantia. GAS, por outro lado, focam-se em tarefas de
processamento do grafo inteiro, tipicamente em ambientes distribuídos, adotando
perdas de qualidade próximas a outras ferramentas NoSQL.

Uma pesquisa recente entre pesquisadores e usuários dessas ferramentas
\cite{sahu} aponta várias questões interessantes sobre seu uso. Primeiramente,
muitos usuários reportam grafos muito grandes, da ordem de milhões de nós e
bilhões de arcos, mesmo em organizações de tamanho médio. Entre pesquisadores
há bastante espaço para o uso de GAS, enquanto que para outros usuários
(por exemplo, na indústria e outras aplicações) é muito mais comum a
preferência por GDBMS, enquanto que GAS não são muito aplicados. Além disso,
entre usuários da indústria é notável a modelagem como grafos de entidades
tradicionalmente usadas em RDBMS, como a tríade
{\ttfamily produto-pedido-transação}. Isso aponta para o valor de análises
realizadas primariamente nas conexões entre tais entidades, possivelmente
complementar às análises relacionais
